\documentclass [a4paper,11pt] {report}
\usepackage{graphicx}
\usepackage{pdfpages}
\usepackage{fancybox}
\usepackage[francais]{babel}
\usepackage[utf8]{inputenc}
\usepackage[T1]{fontenc}
\usepackage{amsmath,amsfonts,amssymb}
\usepackage{fancyhdr}
\usepackage{stackrel}
\usepackage{xspace}
\usepackage{url}
\newcommand{\dsp}{\displaystyle}

\setlength{\parindent}{0pt}
\setlength{\parskip}{1ex}
\setlength{\textwidth}{17cm}
\setlength{\textheight}{24cm}
\setlength{\oddsidemargin}{-.7cm}
\setlength{\evensidemargin}{-.7cm}
\setlength{\topmargin}{-.5in}


%%%%%%
% Pour mise-en-forme des fichiers Ada
%
% voir exemple en fin de ce fichier.
%
% ATTENTION, requiert encoding utf-8 (voir 2ième "\lstset" ci-dessous)
 
\usepackage{listings}
\lstset{
  morekeywords={abort,abs,accept,access,all,and,array,at,begin,body,
      case,constant,declare,delay,delta,digits,do,else,elsif,end,entry,
      exception,exit,for,function,generic,goto,if,in,is,limited,loop,
      mod,new,not,null,of,or,others,out,package,pragma,private,
      procedure,raise,range,record,rem,renames,return,reverse,select,
      separate,subtype,task,terminate,then,type,use,when,while,with,
      xor,abstract,aliased,protected,requeue,tagged,until},
  sensitive=f,
  morecomment=[l]--,
  morestring=[d]",
  showstringspaces=false,
  basicstyle=\small\ttfamily,
  keywordstyle=\bf\small,
  commentstyle=\itshape,
  stringstyle=\sf,
  extendedchars=true,
  columns=[c]fixed
}

% CI-DESSOUS: conversion des caractères accentués UTF-8 
% en caractères TeX dans les listings...
\lstset{
  literate=%
  {À}{{\`A}}1 {Â}{{\^A}}1 {Ç}{{\c{C}}}1%
  {à}{{\`a}}1 {â}{{\^a}}1 {ç}{{\c{c}}}1%
  {É}{{\'E}}1 {È}{{\`E}}1 {Ê}{{\^E}}1 {Ë}{{\"E}}1% 
  {é}{{\'e}}1 {è}{{\`e}}1 {ê}{{\^e}}1 {ë}{{\"e}}1%
  {Ï}{{\"I}}1 {Î}{{\^I}}1 {Ô}{{\^O}}1%
  {ï}{{\"i}}1 {î}{{\^i}}1 {ô}{{\^o}}1%
  {Ù}{{\`U}}1 {Û}{{\^U}}1 {Ü}{{\"U}}1%
  {ù}{{\`u}}1 {û}{{\^u}}1 {ü}{{\"u}}1%
}



\title {{ {\huge Rapport du projet}} \\
``{\em Décomposition en polygones monotones}'' }

\author {Equipe 82 \\
PIERUCCI Dimitri \\ GOUTTEFARDE Léo}
\date{Vendredi 10 Avril 2015}
% \date{Vendredi 10 Avril 2015\endgraf\bigskip
% Equipe NN}

\lhead{Projet d'Algorithmique 2}
\rhead{Rapport}

\begin{document}
\pagestyle{fancy}
\maketitle

\begin{center}
\section* {Introduction }
\end{center}

Nous avons réalisé l'intégralité du sujet, dont pour commencer la conception d'un package d'arbre binaire de recherche générique, que nous avons décidé d'implémenter à l'aider d'un AVL pour optimiser au maximum l'efficacité, ce qui permet ainsi à toutes les opérations demandées (insertion, suppression, recherche) de s'effectuer en $O(log_2(n))$.

L'exécutable décompose sans problème les polygones au format .in en polygones monotones, et effectue également des mises à l'échelle et translations lors de l'export en .svg afin d'obtenir des polygones qui ne débordent pas des dimensions de la figure.

De plus nous avons bien respecté la contrainte de coût au pire cas de $O(h)$ pour l'écriture des fonctions \lstinline!Noeuds_Voisins! et \lstinline!Compte_Position!, en intégrant notamment l'actualisation du champ \lstinline!Compte! des noeuds de l'AVL au sein des opérations de rotation et autres.

Finalement les libérations mémoire sont bien effectuées et il n'y a pas de memory leak.


$\newline$

%Choix / pq
\section* {1\hspace{5mm}Choix d'implémentation }

\subsection* {1.1\hspace{3mm} Structures de données}


\subsection* {1.2\hspace{3mm} Organisation du code}




\section* {2\hspace{5mm}Tests }

\subsection* {2.1\hspace{3mm} Tests unitaires}

\subsection* {2.2\hspace{3mm} Mesures de performance}



\end{document}


